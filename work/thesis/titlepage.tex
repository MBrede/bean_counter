\begin{titlepage}
\newgeometry{left=1.5cm,right=1.5cm,top=3cm,bottom=2cm, bindingoffset=5mm}
	\thispagestyle{empty}
\begin{figure}[H]	%Das große H erzwingt, dass das Bild an genau dieser Position zu sehen ist.
	\centering
	\includegraphics[width = 14cm]{../imgs/FH_Kiel_Logo.png} %Aufpassen! Latex unterscheidet zwischen .png und .PNG
	%Statt "scale = 0.9" empfiehlt sich "width=x cm". Dadurch kontrolliert man die Breite, in der das Bild dargestellt wird.
\end{figure}
	\begin{center}
		\newcommand{\HRule}{\rule{\linewidth}{0.5mm}}
		\HRule \\[0.4cm]
	\begin{spacing}{2.0}
	{\huge Entwicklung eines ML-basierten Tools zur Unterstützung der Bestimmung von Kornverteilungen in elektronenmikroskopischen Aufnahmen}
	\\\vspace*{0.5cm}
	\end{spacing}
	{\Large \textbf{Masterthesis}}\\
	{\Large im Studienfach \glqq{}Data Science\grqq{}}
	\HRule
	\\\vspace*{1.5cm}

	\begin{spacing}{2.0}
	{\Large vorgelegt von:}\\
	{\Large Max Brede}\\\vspace*{1.5cm}
	{\Large betreut von:}\\
	{\Large Prof. Dr. T. Schwörer}\\
	{\Large Prof. Dr. S. Doerfel}
	\\\vspace*{1.5cm}
	\end{spacing}
	\vfill
	\begin{spacing}{0.9}
	% \begin{footnotesize}
	% \begin{minipage}[t]{0.49\textwidth}
	% % \flushleft
	% % \itshape
	% % Master-Thesis\\
	% % \ \\\vspace{2\baselineskip}
	% % 
	% \end{minipage}
	% \begin{minipage}[t]{0.49\textwidth}
	% \flushright
	% \itshape
	% Fachhochschule Kiel\\
	% Hochschule für Angewandte Wissenschaften\\
	% %Fakultät Technik und Informatik\\
	% \vspace{\baselineskip}
	% Fachhochschule Kiel\\
	% University of Applied Sciences\\
	% %Faculty of Engineering and Computer Science\\
	% \end{minipage}
	% \end{footnotesize}
	{\Large Kiel, im Juli 2022}
	\end{spacing}
	\end{center}
\restoregeometry
\end{titlepage}
\newpage
\thispagestyle{empty}
\mbox{}
\newpage